% !TEX TS-program = pdflatex
% !TEX encoding = UTF-8 Unicode

\documentclass[11pt]{article} % use larger type; default would be 10pt

\usepackage[utf8]{inputenc} % set input encoding (not needed with XeLaTeX)

%%% PAGE DIMENSIONS
\usepackage{geometry} % to change the page dimensions
\geometry{letterpaper} % or letterpaper (US) or a5paper or....

\usepackage{graphicx} % support the \includegraphics command and options
\usepackage{circuitikz}

% \usepackage[parfill]{parskip} % Activate to begin paragraphs with an empty line rather than an indent

%%% PACKAGES
\usepackage{booktabs} % for much better looking tables
\usepackage{array} % for better arrays (eg matrices) in maths
\usepackage{paralist} % very flexible & customisable lists (eg. enumerate/itemize, etc.)
\usepackage{verbatim} % adds environment for commenting out blocks of text & for better verbatim
\usepackage{subfig} % make it possible to include more than one captioned figure/table in a single float
\usepackage{hyperref}
\usepackage{enumitem}
\usepackage{xcolor}
% These packages are all incorporated in the memoir class to one degree or another...

%%% HEADERS & FOOTERS
\usepackage{fancyhdr} % This should be set AFTER setting up the page geometry
\pagestyle{fancy} % options: empty , plain , fancy
\renewcommand{\headrulewidth}{0pt} % customise the layout...
\lhead{}\chead{}\rhead{}
\lfoot{}\cfoot{\thepage}\rfoot{}

%%% SECTION TITLE APPEARANCE
\usepackage{sectsty}
\allsectionsfont{\sffamily\mdseries\upshape} % (See the fntguide.pdf for font help)
% (This matches ConTeXt defaults)

\hypersetup{
    colorlinks=true,
    linkcolor=blue,
    filecolor=magenta,      
    urlcolor=cyan,
}

%%% ToC (table of contents) APPEARANCE
\usepackage[nottoc,notlof,notlot]{tocbibind} % Put the bibliography in the ToC
\usepackage[titles,subfigure]{tocloft} % Alter the style of the Table of Contents
\renewcommand{\cftsecfont}{\rmfamily\mdseries\upshape}
\renewcommand{\cftsecpagefont}{\rmfamily\mdseries\upshape} % No bold!

%%% END Article customizations

%%% The "real" document content comes below...

\title{Minimal 74-Series Computer}
\author{Matthew James Bellafaire}
%\date{} % Activate to display a given date or no date (if empty),
         % otherwise the current date is printed 

\begin{document}

\maketitle

\tableofcontents
\listoffigures

\section{Purpose}

The purpose of this project is to design and implement a VHDL and physical minimal computer for use as a display demonstration. 
The design is created and validated inside of Vivado 2020 with 74-series logic that was manually modeled. 

\subsection{Functions And Instructions}



\subsection{Flags and Registers}
Operational flags are shown in Table~\ref{tabFlags}, all flags are utilized for operational purposes in conditional operations inside the CPU.

\begin{table}
	\centering
\begin{tabular}{| c | c | c |}
\hline
Flag Register & Bit in flags register & Description \\
\hline
C & 0 & carry flag \\
Z & 1 & zero flag \\
L & 2 & A less than B flag \\
G & 3 & A greater than B flag \\
D& 4& Done\\
\hline
\end{tabular}
\caption{Table of bits in flag register}
\label{tabFlags}
\end{table}
\section{Component Selection}
This computer will be built primarily out of 74 series logic due to the wide availablity and versatility of the series. 
This section details components selected for the final design of this computer.
\begin{description}[font=$\bullet$~\normalfont\scshape\color{red!50!black}]
\vspace{-5mm}
\item[CD74HCT283] - standard 4-bit adder IC selected for use in the ALU \\\vspace{-5mm}
\item[SN74HCT574] - octal edge-triggered D-type flipflop with 3 state output. General purpose register memory, will be used in registers A, B, PC, and PD \\\vspace{-5mm}
\item[AT28C64B] - Parallel EEPROM (8K x 8). General purpose program memory, will be where the program and varaibles are stored \\\vspace{-5mm}
\item[SN74HCT08] - quadruple 2 input AND gate IC, selected for use in the ALU \\\vspace{-5mm}
\item[SN74HCT32] - Quadruple 2-input OR Gate IC, selected for use in the ALU \\\vspace{-5mm}
\item[74HCT163] - Presettable synchronous 4-bit binary counter; synchronous reset. Primary program counter \\\vspace{-5mm}
\item[74HCT85] - 4-bit magnitude comparator. utilized in CPU to greater than, less than, and equals flag registers \\\vspace{-5mm}
\item[74HCT245] - Octal bus tranciever with 3-state outputs. used to switch a device onto or off of the system bus \\\vspace{-5mm}
\item[74HCT04] - hex inverter \\\vspace{-5mm}
\item[74HCT138] - 3 to 8 decoder with inverting outputs \\\vspace{-5mm}
\item[74HCT238] - 3 to 8 decoder with non-inverting outputs \\\vspace{-5mm}
\end{description}

\end{document}
